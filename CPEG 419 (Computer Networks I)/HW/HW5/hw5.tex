\documentclass{article}
\usepackage[utf8]{inputenc}
\usepackage[margin=0.4in]{geometry}


\title{Computer Networks HW 5}
\author{Shane Cincotta }
\date{May 11, 2020}

\usepackage{natbib}
\usepackage{graphicx}

\begin{document}

\maketitle

\section*{Chapter 4 Problem 1}
\subsection*{a}
\begin{tabular}{ |l|l| }
  \hline
  Destination Address & The Interface \\ \hline
  H3 & Link Interface 3 \\ \hline
\end{tabular}
\subsection*{b}
Implementing a forwarding table which has more than 1 route which leads to the same destination is impossible.  The reason is because router A can either have an entry to H3 through link interface 3 or 4.\\

\section*{Chapter 4, Problem 5}
\subsection*{a}
\begin{tabular}{ |l|l| }
  \hline
  Prefix Match & Link Interface \\ \hline
  1110000000 & 0 \\ \hline
  1110000001000000 & 1 \\ \hline
  1110000 & 2 \\ \hline
  111000011 & 3 \\ \hline
  Otherwise & 4 \\ \hline
\end{tabular}

\subsection*{b}
According to my table, the prefix match for the first address is the 5th entry which maps to link interface 3.  The prefix match for the second address is the 3rd entry which maps to link interface 2.  The prefix match for the 3rd address is the 4th entry, which maps to link interface 3.\\

\section*{Chapter 4, Problem 6}
\begin{tabular}{ |l|l| }
  \hline
  Destination Address Range & Link Interface \\ \hline
  00000000 through 00111111 & 0 \\ \hline
  01000000 through 01011111 & 1 \\ \hline
  01100000 through 01111111 & 2 \\ \hline
  10000000 through 10111111 & 2 \\ \hline
  11000000 through 11111111 & 3 \\ \hline
\end{tabular}
\begin{tabular}{ |l|l| }
  \hline
  Number of Addresses & Link Interface \\ \hline
   $2^6$ & 0 \\ \hline
   $2^5$ & 1 \\ \hline
   $2^5 + 2^6$ & 2 \\ \hline
   $2^6$ & 3 \\ \hline
\end{tabular}

\section*{Chapter 4, Problem 8}
\subsection*{Subnet 2}
Subnet 2 has at least 90 interfaces, which corresponds to 7 bits.  Thus the prefix range is 32-7=25.  Therefore subnet 2 consists of the address which ranges from 223.1.17.0/25 to223.1.17.127/25 

\subsection*{Subnet 1}
Subnet 1 has at least 60 interfaces, which corresponds to 6 bits.  Thus the prefix range is 32-6=26.  Therefore subnet 1 consists of the address which ranges from 223.1.17.128/26 to223.1.17.191/26 

\subsection*{Subnet 3}
Subnet 3 has at least 12 interfaces, which corresponds to 4 bits.  Thus the prefix range is 32-4=28.  Therefore subnet 3 consists of the address which ranges from 223.1.17.192/28 to223.1.17.207/28.

\section*{Chapter 4, Problem 10}
\begin{tabular}{ |l|l|l| }
  \hline
  Destination Address & a.b.c.d/x notation & Link Interface\\ \hline
   1110000000 & 224.0.0.0/10 & 0 \\ \hline
   1110000001000000 & 224.64.0.0/16 & 1\\ \hline
   1110000 & 224.0.0.0/7 & 2\\ \hline
   111000011 & 226.128.0.0/9 & 3\\ \hline
   Otherwise & 0.0.0.0/0 & 3\\ \hline
\end{tabular}

\section*{Chapter 4, Problem 12}
\subsection*{a}
\begin{tabular}{ |l|l|l| }
  \hline
  Subnet & IP Address & Available UP Addresses\\ \hline
    Subnet A & 214.97.254.0/24 & 254 \\ \hline
    Subnet B & 214.97.254.0/25 - 214.97.255.0/29 & 120 \\ \hline
    Subnet C & 214.97.254.128/25 & 128 \\ \hline
    Subnet D & 214.97.254.0/30 & 2 \\ \hline
    Subnet E & 213.97.254.2/31 & 2 \\ \hline
    Subnet E & 213.97.254.3/31 & 4 \\ \hline
\end{tabular}

\subsection*{b}
Forwarding table for router 1:\\
\begin{tabular}{ |l|l|l| }
  \hline
  Longest Prefix & Link Interface & Action \\ \hline
     11010110.01100001.1111111 & A & Forwards to subnet A \\ \hline
     11010110.01100001.11111110.0000010 & F & Forwards to router 2\\ \hline
     11010110.01100001.11111110.0000000 & D & Forwards to router 3\\ \hline
\end{tabular}

Forwarding table for router 2:\\
\begin{tabular}{ |l|l|l| }
  \hline
  Longest Prefix & Link Interface & Action \\ \hline
     11010110.01100001.11111110.1 & C & Forwards to subnet C \\ \hline
     11010110.01100001.11111110.0000001 & E & Forwards to router 3 \\ \hline
     11010110.01100001.11111110.0000010 & F & Forwards to router 1\\ \hline
\end{tabular}

Forwarding table for router 3:\\
\begin{tabular}{ |l|l|l| }
  \hline
  Longest Prefix & Link Interface & Action \\ \hline
     11010110.01100001.11111110.0 & B & Forwards to subnet B\\ \hline
     11010110.01100001.11111110.0000000 & D & Forwards to router 1\\ \hline
     11010110.01100001.11111110.0000001 & E & Forwards to router 2\\ \hline
\end{tabular}

\section*{Chapter 4, Problem 14}
In each fragment, the max size of the data field is MTU - IP Header = 700 - 20 = 680.\\
\newline The number of fragments = $\frac{Datagram - Header}{MTU - Header}$ = $\frac{2400-20}{680}$ = 4.\\
\newline 
\begin{tabular}{ |l|l|l|l|l|l|l| }
  \hline
  Fragment Number & Data Size & IP Header Size & Fragment Size & ID & Offset & Flag Value \\ \hline
      1 & 680 & 20 & 700 & 422 & 0 & 1\\ \hline
      2 & 680 & 20 & 700 & 422 & 85 & 1\\ \hline
      3 & 680 & 20 & 700 & 422 & 170 & 1\\ \hline
      4 & 340 & 20 & 360 & 422 & 255 & 0\\ \hline
\end{tabular}

\section*{Chapter 4, Problem 16}
\subsection*{a}
\begin{tabular}{ |l|l| }
  \hline
  Interface Number & IP Address \\ \hline
      1 & 192.168.1.1 \\ \hline
      2 & 192.168.1.3 \\ \hline
      3 & 192.168.1.2 \\ \hline
\end{tabular}
\newline
The network router interface hs 192.168.1.4\\

\subsection*{b}
\begin{tabular}{ |l|l| }
  \hline
  WAN Side & LAN Side \\ \hline
      23.34.112.235,4000 & 192.168.1.1,3345 \\ \hline
      23.34.112.235,4001 & 192.168.1.1,3346 \\ \hline
      23.34.112.235,4002 & 192.168.1.2,3445 \\ \hline
      23.34.112.235,4003 & 192.168.1.2,3446 \\ \hline
      23.34.112.235,4004 & 192.168.1.3,3545 \\ \hline
      23.34.112.235,4005 & 192.168.1.3,3546 \\ \hline
\end{tabular}

\section*{Chapter 4, Problem 19}
\begin{tabular}{ |l|l| }
  \hline
  Match & Action \\ \hline
      Port = 1; IP Source = 10.3.x.x ; IP Destination = 10.1.x.x & Forward to 2 \\ \hline
      Port = 2; IP Source = 10.1.x.x ; IP Destination = 10.3.x.x & Forward to 1 \\ \hline
      Port = 1; IP Destination = 10.2.0.3 & Forward to 3 \\ \hline
      Port = 2; IP Destination = 10.2.0.3 & Forward to 3 \\ \hline
      Port = 1; IP Destination = 10.2.0.4 & Forward to 4 \\ \hline
      Port = 2; IP Destination = 10.2.0.4 & Forward to 4 \\ \hline
      Port = 4 & Forward to 3 \\ \hline
      Port = 3 & Forward to 4 \\ \hline
\end{tabular}

\section*{Chapter 4, Problem 20}
\begin{tabular}{ |l|l|l|l|l|l|l| }
  \hline
  Source Host & Destination Host & Source IP & Destination IP & Port Number & Interface Number & Action \\ \hline
      H3 & H1 & 10.2.0.3 & 10.1.0.1 & 3 & 2 & Forward \\ \hline
      H3 & H2 & 10.2.0.3 & 10.1.0.2 & 3 & 2 & Forward \\ \hline
      H3 & H5 & 10.2.0.3 & 10.3.0.5 & 3 & 2 & Forward \\ \hline
      H3 & H6 & 10.2.0.3 & 10.3.0.6 & 3 & 2 & Forward \\ \hline
\end{tabular}
\newline \\
\begin{tabular}{ |l|l|l|l|l|l|l| }
  \hline
  Source Host & Destination Host & Source IP & Destination IP & Port Number & Interface Number & Action \\ \hline
      H4 & H1 & 10.2.0.4 & 10.1.0.1 & 4 & 1 & Forward \\ \hline
      H4 & H2 & 10.2.0.4 & 10.1.0.2 & 4 & 1 & Forward \\ \hline
      H4 & H5 & 10.2.0.4 & 10.3.0.5 & 4 & 1 & Forward \\ \hline
      H4 & H6 & 10.2.0.4 & 10.3.0.6 & 4 & 1 & Forward \\ \hline
\end{tabular}

\end{document}